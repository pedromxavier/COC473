\documentclass{homework}
\usepackage{homework}

\title{COC473 - Lista 4}
\author{Pedro Maciel Xavier}
\register{116023847}
\notes{\textbf{Nota:} Na primeira seção da Lista estão os trechos de código dos programas pedidos. Na segunda parte, estão os resultados dos programas assim como a análise destes. Por fim, no apêndice está o código completo. Caso os gráficos estejam pequenos, você pode ampliar sem problemas pois foram renderizados diretamente no formato \texttt{.pdf}.}

\newcommand{\vecx}[3]{
	\ensuremath{\left[\begin{array}{@{}c@{}}
			#1\\
			#2\\
			#2
		\end{array}\right]}
}

\begin{document}
	
	\maketitle
	
	\section*{Programas}
	
	\questx[Bisseção]%%1
	
	\lstinputlisting[firstline=107, lastline=151, style=fortranstyle, gobble=0]{../src/calclib.f95}
	
	\quest[Método de \textit{Newton}]%%2
	
	\subquest[Original]%%a
	
	\lstinputlisting[firstline=153, lastline=201, style=fortranstyle, gobble=0]{../src/calclib.f95}
	
	\subquest[Secante]%%a
	
	\lstinputlisting[firstline=203, lastline=252, style=fortranstyle, gobble=0]{../src/calclib.f95}

	\quest[Método de Interpolação Inversa]%%4
	
	\lstinputlisting[firstline=254, lastline=312, style=fortranstyle, gobble=0]{../src/calclib.f95}
	
	\quest[Sistemas de equações]%%4
	
		\subquest[Método de \textit{Newton} (Derivadas Parciais Analíticas)]%%a
		
		\lstinputlisting[firstline=314, lastline=375, style=fortranstyle, gobble=0]{../src/calclib.f95}
		
		\subquest[Método de \textit{Newton} (Derivadas Parciais Numéricas)]%%a
		
		\lstinputlisting[firstline=377, lastline=439, style=fortranstyle, gobble=0]{../src/calclib.f95}
	
		\subquest[Método de \textit{Broyden}]%%a
		
		\lstinputlisting[firstline=441, lastline=497, style=fortranstyle, gobble=0]{../src/calclib.f95}
	
	\quest[Ajuste de curvas não-lineares]%%5
	
	\lstinputlisting[firstline=499, lastline=564, style=fortranstyle, gobble=0]{../src/calclib.f95}
	
	\setcounter{quests}{0}
	
	\newpage
	\section*{Aplicações}
	
	\questx[Utilizando os programas desenvolvidos encontre as raízes da seguinte equação por todos os métodos apresentados em sala de aula.
	$$f(x) = \log\left(\cosh\left(x \sqrt{g k}\right)\right) - 50$$
	onde $g = 9.806$ e $k = 0.00341$.]
	
	\lstinputlisting[firstline=1, lastline=17, style=blankstyle]{answer/L4.txt}
	
	\quest[Repita o exercício anterior para a função:
	$$f(x) = 4 \cos(x) - e^{2 x}$$]
	
	\lstinputlisting[firstline=19, lastline=35, style=blankstyle]{answer/L4.txt}
	
	\quest[Encontre uma solução para o seguinte sistema de equações não-lineares pelos métodos de \textit{Newton} e \textit{Broyden} utilizando os programas desenvolvidos.
	\begin{align*}
	&16x^4 + 16 y^4 + z^4 = 16\\
	&x^2 + y^2 + z^2 = 3\\
	&x^3 - y + z = 1
	\end{align*}]

	\lstinputlisting[firstline=37, lastline=87, style=blankstyle]{answer/L4.txt}
	
	\quest[Resolva, utilizando os programas desenvolvidos, o seguinte sistema de equações não-lineares (usando os Métodos de \textit{Newton} e \textit{Broyden}):
	{\normalsize%% 
	\begin{align*}
	&2 c_3^2 + c_2^2 + 6 c_4^2 = 1\\
	&8 c_3^3 + 6 c_3 c_3^3 + 36 c_3 c_2 c_4 + 108 c_3 c_4^2 = \theta_1\\
	&60 c_3^4 + 60 c_3^2 c_2^2 + 576 c_3^2 c_2 c_4 + 2232 c_3^2 c_4^2 + 252 c_4^2 c_2^2 + 1296 c_4^3 c_2 + 3348 c_4^4 + 24 c_2^3 c_4 + 3 c_2 = \theta_2
	\end{align*}%%
	}%%
	considerando os seguintes casos:]%%
	{\Large\begin{enumerate}[label=\alph*)]%%
		\item $\theta_1 = 0.00$ e $\theta_2 = 3.0$;%%
		\item $\theta_1 = 0.75$ e $\theta_2 = 6.5$;%%
		\item $\theta_1 = 0.00$ e $\theta_2 = 11.667$;%%
	\end{enumerate}}

	\lstinputlisting[firstline=89, lastline=241, style=blankstyle]{answer/L4.txt}
	
	\quest[Utilizando o programa desenvolvido, ajuste uma função do tipo $f(x) = b_0 + b_1 x^{b_2}$ ao conjunto de dados abaixo:]%% 5
	{\Large	\begin{center}
		\begin{tabular}{|c|c|c|c|}
			\hline
			$x$ & 1 & 2 & 3 \\
			\hline
			$y$ & 1 & 2 & 9 \\
			\hline
		\end{tabular}
	\end{center}}

	
	\lstinputlisting[firstline=243, lastline=284, style=blankstyle]{answer/L4.txt}

	\pagebreak
	\appendixpage
	\appendix \section*{Código - Programa Principal}
	\lstinputlisting[style=fortranstyle, gobble=0]{../src/main4.f95}
	\appendix \section*{Código - Definição das Funções}
	\lstinputlisting[style=fortranstyle, gobble=0]{../src/funcmod.f95}
	\appendix \section*{Código - Métodos Numéricos}
	\lstinputlisting[style=fortranstyle, gobble=0]{../src/calclib.f95}
	\appendix \section*{Código - Métodos com Matrizes}
	\lstinputlisting[style=fortranstyle, gobble=0]{../src/matrixlib.f95}
	\appendix \section*{Código - Biblioteca Auxiliar}
	\lstinputlisting[style=fortranstyle, gobble=0]{../src/utillib.f95}
%	\begin{thebibliography}{10}
%		
%	\end{thebibliography}
\end{document}
